% Replace Replace with Second Chapter Name
% Replace c2_secondchapter:cha with your chapter title label (no spaces, only lower case letters)
% Replace the text below \end{chapterpage} and insert your own text.

\begin{chapterpage}{Replace with Second Chapter Name}{c2_secondchapter:cha}

\begin{myquotation} The perfect place for an introducing quotation.\par\vspace*{15mm}
\mbox{}\hfill \emdash{}Famous Person\index{Person, Famous}
% Add the source.
%, \citetitle{bibitem}\index{@\citetitle{bibitem}} \ifxetex\label{famousperson-bibitem-quote}\else\citep[p.~123]{bibitem}\fi
\par\end{myquotation}

\end{chapterpage}

% -------------------- replace or remove text below and paste your own text ------


\section{Images}\label{c1_images:sec}

As in Word, in \textit{LaTeX}, images are separate from the text. Images are usually packaged together with a caption and a label to reference it from the text. These three entities are packaged together into a figure. The figure itself configures the size of the image as well as where it should be put. Let us look at a code sample:
\begin{lstlisting}
\begin{figure}[H]
	\centering
	\ifxetex
		\adjustbox{max width=.95\columnwidth, max height=.4\textheight}{
			\includegraphics{images/ebookLatex_Cover.png}
		}
	\else
		\includegraphics{images/ebookLatex_Cover.jpg}
	\fi
	\caption{The cover of this book.}
	\label{c1_cover:fig}
\end{figure}
\end{lstlisting}

Let us go through this line by line. At the core is the image, included with \textbf{\textbackslash includegraphics\{path to file\}}. It inserts the image specified by the ``path to file.'' In the example, there are two versions of \textbackslash includegraphics, one for the print output (\textbackslash ifxetex), one for the e-book. This way, we include either the larger, lossless PNG version, or a smaller, lossy JPG version of the cover. For the e-book, we do not need to provide any image size as the first output will be a HTML file with no boundaries. For the print edition, we have a clear boundary, namely the page width. With the \textbf{\textbackslash adjustbox} command, we can adjust the image size according to the page width (\textbackslash columnwidth) and page height (\textbackslash textheight). 

In this case, in order to align with the caption and be able to show two images on one page, we want an image to be generally at most 95\% of the page width and 40\% of the page height.

Below this \textbackslash ifxetex - \textbackslash else - \textbackslash fi construct, there is the caption and the label. \textit{LaTeX} automatically numbers each figure, so in the text, we can later refer to it with ``\textbackslash ref\{c1\_cover:fig\}'' which prints out the number of the figure. Finally, all these commands are centered with the \textbf{\textbackslash centering} command and surrounded with the figure environment. The \textbf{[H]} instructs \textit{LaTeX} to try to place the image exactly where it is in the \textit{LaTeX} code. 

\begin{figure}[H]
	\centering
	\ifxetex
		\adjustbox{max width=.95\columnwidth, max height=.4\textheight}{
			\includegraphics{images/ebookLatex_Cover_highres.png}
		}
	\else
		\includegraphics{images/ebookLatex_Cover.jpg}
	\fi
	\caption{The cover of this book.}
	\label{c1_cover:fig}
\end{figure}

In Figure~\ref{c1_cover:fig}, you can see the result of the command.

Instead of graphics, you can also include other TEX files that contain graphics (or commands to draw graphics, see chapter~\ref{c1_tikzgraphics:sec}).


\section{TikZ Graphics}\label{c1_tikzgraphics:sec}

For graphics, you can use the inbuilt TikZ graphics generator. Due to its flexibility, I even recommend images you already have for a number of reasons:

\begin{itemize}
\item TikZ graphics can very easily changed (especially for for example translations or making corrections).
\item TikZ graphics are small and flexible. They can be easily scaled to any size and are directly integrated into your project (no time-consuming editing in an external graphics program necessary).
\item TikZ graphics look better. As vector graphics are sent directly to the printer, we need not to worry about readability.
\end{itemize}

If you want to create a TikZ graphic, simply create a new TEX file in the \textit{tex-images} folder and include it with \textbf{\textbackslash input} (replacing \textbackslash includegraphics) where you want to. 

Then, do a ``recompile from scratch'' by clicking on the top right corner of the preview window (showing Warning or Error) to regenerate the TikZ file. If ``up-to-date and saved'' is shown, delete the \textit{tikz-cache} directory and recreate it. This deletes the generated cached image files that do not show up in \textit{Overleaf} itself.

For the format of the file itself, it is a series of commands surrounded by the \textbf{\textbackslash begin\{tikzpicture\} \dots \textbackslash end\{tikzpicture\}} environment. Discussing all the commands is beyond the scope of this book, so I recommend three options:

\begin{itemize}
\item Check out the PGF manual at \url{https://www.ctan.org/pkg/pgf}. It is more than 1100 pages full with documentation of each command and corresponding examples.
\item Check out the few example TikZ pictures from my two books \citetitle{PFH1E} and \citetitle{PFH2E} in the \textit{tex-images} directory.
\item Send us descriptions / drafts of the graphics you need. We can help you at affordable rates. Contact us at mail@lode.de.
\end{itemize}