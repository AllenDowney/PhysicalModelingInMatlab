

\section{Glossary}

\begin{description}

\item[interpreter:] The program that reads and executes MATLAB code.

\item[command:] A line of MATLAB code executed by the interpreter.

\item[prompt:] The symbols the interpreter prints to indicate that it's
waiting for you to type a command.

\item[operator:] One of the symbols, like \lstinline{*} and \lstinline{+}, that
represent mathematical operations.

\item[operand:] A number or variable that appears in an expression along
with operators.

\item[expression:] A sequence of operands and operators that specifies
a mathematical computation and yields a value.

\item[value:] The numerical result of a computation.

\item[evaluate:] To compute the value of an expression.

\item[order of operations:] The rules that specify which operations
in an expression are performed first.

\item[function:] A named computation; for example {\tt log10} is the
name of a function that computes logarithms in base 10.

\item[call:] To cause a function to execute and compute a result.

\item[function call:] A kind of command that executes a function.

\item[argument:] An expression that appears in a function call to
specify the value the function operates on.

\item[nested function call:] An expression that uses the result from
one function call as an argument for another.

\item[variable:] A named value.

\item[assignment statement:] A command that creates a new variable
(if necessary) and gives it a value.

\item[workspace:] A set of variables and their values.

\item[string:] A value that consists of a sequence of characters (as
opposed to a number).


\end{description}



\section{Glossary}

\begin{description}

\item[M-file:] A file that contains a MATLAB program.

\item[script:] An M-file that contains a sequence of MATLAB commands.

\item[search path:] The list of folders where MATLAB looks for
M-files.

\item[precondition:] Something that must be true when the script
starts, in order for it to work correctly.

\item[postcondition:] Something that will be true when the script
completes.

\item[target:] The variable on the left side of an assignment statement.

\item[floating-point:] A way to represent numbers in a computer.

\item[scientific notation:] A format for typing and displaying large
and small numbers; e.g. {\tt 3.0e8}, which represents $3.0 \times 10^8$
or 300,000,000.

\item[comment:] Part of a program that provides additional information
about the program, but does not affect its execution.

\end{description}


\section{Glossary}

\begin{description}

\item[absolute error:] The difference between an approximation and
an exact answer.

\item[relative error:] The difference between an approximation and
an exact answer, expressed as a fraction or percentage of the exact
answer.

\item[loop:] A part of a program that runs repeatedly.

\item[loop variable:] A variable, defined in a {\tt for} statement,
that gets assigned a different value each time through the loop.

\item[range:] The set of values assigned to the loop variable, often
specified with the colon operator; for example {\tt 1:5}.

\item[body:] The statements inside the for loop that are run
repeatedly.

\item[sequence:] In mathematics, a set of numbers that correspond
to the positive integers.

\item[element:] A member of the set of numbers in a sequence.

\item[recurrently:] A way of computing the next element of a sequence
based on previous elements.

\item[directly:] A way of computing an element in a sequence without
using previous elements.

\item[series:] The sum of the elements in a sequence.

\item[accumulator:] A variable that is used to accumulate a result
a little bit at a time.

\item[generalization:] A way to make a program more versatile, for
example by replacing a specific value with a variable that can have
any value.

\item[incremental development:] A way of programming by making a series
of small, testable changes.

\item[scaffolding:] Code you write to help you program or debug, but
which is not part of the finished program.

\end{description}


\section{Glossary}

\begin{description}

\item[compound statement:] A statement, like {\tt if} and {\tt for}, that
contains other statements in an indented body.

\item[nesting:] Putting one compound statement in the body of another.

%\item[relational operator:] An operator that compares two values and generates a logical value as a result.

%\item[logical value:] A value that represents either ``true'' or ``false''.  MATLAB uses the values 1 and 0, respectively.

% \item[flag:] A variable that contains a logical value, often used to store the status of some condition.

\item[scalar:] A single value.

\item[vector:] A sequence of values.

\item[matrix:] A two-dimensional collection of values (also called
``array'' in some MATLAB documentation).

\item[index] An integer value used to indicate one of the values
in a vector or matrix (also called subscript in some MATLAB documentation).

\item[element:] One of the values in a vector or matrix.

\item[elementwise:] An operation that acts on the individual elements
of a vector or matrix (unlike some linear algebra operations).

\item[reduce:] A way of processing the elements of a vector and
generating a single value; for example, the sum of the elements.

\item[apply:] A way of processing a vector by performing some operation
on each of the elements, producing a vector that contains the
results.

\item[search:] A way of processing a vector by examining the
elements in order until one is found that has the desired property.

\item[name collision:] The scenario where two scripts that use the
same variable name interfere with each other.

\end{description}


\section{Glossary}

\begin{description}

\item[side-effect:] An effect, like modifying the workspace, that
is not the primary purpose of a script.

\item[input variable:] A variable in a function that gets its value,
when the function is called, from one of the arguments.

\item[output variable:] A variable in a function that is used to
return a value from the function to the caller.

\item[signature:] The first line of a function definition, which
specifies the names of the function, the input variables and the
output variables.

\item[silent function:] A function that doesn't display anything
or generate a figure, or have any other side-effects.

\item[logical function:] A function that returns a logical value
(1 for ``true'' or 0 for ``false'').

\item[encapsulation:] The process of wrapping part of a program in
a function in order to limit interactions (including name collisions)
between the function and the rest of the program.

\item[generalization:] Making a function more versatile by replacing
specific values with input variables.

\item[abstraction:] The process of ignoring the details of how
a function works in order to focus on a simpler model of what the
function does.

\end{description}



\section{Glossary}

\begin{description}

\item[analytic solution:] A way of solving an equation by performing
algebraic operations and deriving an explicit way to
compute a value.

\item[numerical solution:] A way of solving an equation by finding
a numerical value that satisfies the equation, often approximately.

\item[numerical method:] A method (or algorithm) for generating
a numerical solution.

\item[zero (of a function):] An argument that makes the result of a function $0$.

\item[function handle:] In MATLAB, a function handle is a way of
referring to a function by name (and passing it as an argument)
without calling it.

\item[shadow:] A kind of name collision in which a new definition
causes an existing definition to become invisible.  In MATLAB,
variable names can shadow built-in functions (with hilarious results).

\end{description}


\section{Glossary}

\begin{description}

\item[existential quantification:] A logical condition that expresses the idea that ``there exists'' an element of a set with a certain property.

\item[universal quantification:] A logical condition that expresses
the idea that all elements of a set have a certain property.

\item[logical vector:] A vector, usually the result of applying a logical operator to a vector, that contains logical values 1 and 0.

\end{description}


\section{Glossary}

\begin{description}

\item[top-level function:]  The first function in an M-file;
it's the only function that can be called from the Command
Window or from another file.

\item[helper function:] A function in an M-file that is not
the top-level function; it only be called from another function
in the same file.

\item[differential equation (DE):] An equation that relates the
derivatives of an unknown function.

\item[ordinary DE (ODE):] A DE in which all derivatives are taken with
respect to the same variable.

\item[partial DE (PDE):] A DE that includes derivatives with respect to more than one variable

\item[first-order DE:] A DE that includes only first derivatives.

\item[linear DE:] A DE that includes no products or powers of the
function and its derivatives.

\item[time step:] The interval in time between successive estimates
in the numerical solution of a DE.

\item[parameter:] A value that appears in a model to quantify some
physical aspect of the scenario being modeled.

\end{description}



\section{Glossary}

\begin{description}

\item[row vector:] A matrix that has only one row.

\item[column vector:] A matrix that has only one column.

\item[transpose:] An operation that transforms the rows of a matrix
into columns (or the other way around, if you prefer).

\item[system of equations:] A collection of equations written in terms of
the same set of variables.

\item[unpack:] To copy the elements of a vector into a set of variables.

\item[pack:] To copy values from a set of variables into a vector.

\item[state:] If a system can be described by a set of variables,
the values of those variables are called the state of the system.

\item[phase plot:] A plot that shows the state of a system as point
in the space of possible states.

\item[trajectory:] A path in a phase plot that shows how the state of
a system changes over time.


\end{description}


\section{Glossary}

\begin{description}

\item[spatial vector:] A value that represents a
multidimensional physical quantity like position, velocity,
acceleration or force.

\item[unit vector:] A vector with norm 1, used to indicate
direction.

\item[norm:] The magnitude of a vector.  Sometimes called ``length,''
but not to be confused with the number of elements in a MATLAB
vector.

\item[concatenation:] The operation of joining two vectors or matrices end-to-end to
form a new vector or matrix.

\end{description}


\section{Glossary}

\begin{description}

\item[dot product:] A scalar product of two vectors, proportional to the
norms of the vectors and the cosine of the smallest angle between them.

\item[cross product:] A vector product of two vectors with norm
proportional to the norms of the vectors and the sine of the angle
between them, and direction perpendicular to both.

\item[projection:] The component of one vector that is in the
direction of the other (might be used to mean ``scalar projection'' or
``vector projection'').

\end{description}

