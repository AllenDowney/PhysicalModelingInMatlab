

\chapter*{Glossary}

\begin{description}


\item[Absolute error] The difference between an approximation and
an exact answer.

\item[Abstraction] The process of ignoring the details of how
a function works in order to focus on a simpler model of what the
function does.

\item[Accumulator] A variable that is used to accumulate a result
a little bit at a time.

\item[Analytic solution] A way of solving an equation by performing
algebraic operations and deriving an explicit way to
compute a solution.

\item[Apply] A way of processing a vector by performing some operation
on each of the elements, producing a vector that contains the
results.

\item[Argument] An expression that appears in a function call to
specify the value the function operates on.

\item[Assignment statement] A command that creates a new variable
(if necessary) and gives it a value.

\item[Body] The statements inside a loop that are run
repeatedly.

\item[Call (a function)] To cause a function to execute and compute a result.

\item[Column vector] A matrix that has only one column.

\item[Command] A line of MATLAB code executed by the interpreter.

\item[Comment] Part of a program that provides additional information
about the program but does not affect its execution.

\item[Compound statement] A statement, like \lstinline{if} and \lstinline{for}, that
contains other statements in an indented body.

\item[Concatenation] The operation of joining two vectors or matrices end-to-end to
form a new vector or matrix.

%\item[Cross product] A product of two vectors with norm
%proportional to the norms of the vectors and the sine of the angle
%between them, and direction perpendicular to both.

\item[Differential equation (DE)] An equation that relates the
derivatives of an unknown function.

\item[Directly (compute)] A way of computing an element in a sequence without
using previous elements.

%\item[dot product:] A scalar product of two vectors, proportional to the
%norms of the vectors and the cosine of the smallest angle between them.


\item[Element (of a matrix)] One of the values in a vector or matrix.

\item[Element (of a sequence)] One of the numbers in a mathematical \mbox{sequence}.

\item[Elementwise] An operation that acts on the elements
of a vector or matrix (unlike some linear algebra operations).

\item[Encapsulation] The process of wrapping part of a program in
a function in order to limit interactions (including name collisions)
between the function and the rest of the program.

\item[Evaluate] To compute the value of an expression.

\item[Existential quantification] A logical condition that expresses the idea that ``there exists'' an element of a set with a certain property.

\item[Expression] A sequence of operands and operators that specifies
a mathematical computation and yields a value.

\item[First-order DE] A differential equation that includes only first derivatives.

\item[Floating-point] A way to represent numbers in a computer.

\item[Function] A named computation; for example, \lstinline{log10} is the
name of a function that computes logarithms in base 10.

\item[Function call] A command that executes a function.

\item[Function handle] A function handle is a way of
referring to a function by name (and passing it as an argument)
in MATLAB without calling it.

\item[Generalization] Making a function more versatile by replacing
specific values with input variables.

\item[Helper function] A function in an M-file that is not
the top-level function; it can only be called from another function
in the same file.

\item[Incremental development] A way of programming by making a series
of small, testable changes.

\item[Index] An integer value used to indicate one of the values
in a vector or matrix (also called subscript in some MATLAB documentation).

\item[Input variable] A variable in a function that gets its value,
when the function is called, from one of the arguments.

\item[Interpreter] The program that reads and executes MATLAB code.

\item[Linear DE] A differential equation that includes no products or powers of the
function and its derivatives.

\item[Logical function] A function that returns a logical value
(1 for ``true'' or 0 for ``false'').

\item[Logical vector] A vector, often the result of applying a logical operator to a vector, that contains logical values 1 and 0.

\item[Loop] A part of a program that runs repeatedly.

\item[Loop variable] A variable, defined in a loop,
that gets assigned a different value each time through the loop.

\item[M-file] A file that contains a MATLAB program.

\item[Matrix] A two-dimensional collection of values (also called
``array'' in some MATLAB documentation).

\item[Name collision] The scenario where two scripts that use the
same variable name interfere with each other.

\item[Nested function call] An expression that uses the result from
one function call as an argument for another.

\item[Nesting] Putting one compound statement in the body of another.

\item[Norm] The magnitude of a vector, sometimes called ``length,''
but not to be confused with the number of elements in a MATLAB
vector.

\item[Numerical method] A method (or algorithm) for generating
a numerical solution.

\item[Numerical solution] A way of solving an equation by finding
a numerical value that satisfies the equation, often approximately.

\item[Operand] A number or variable that appears in an expression along
with operators.

\item[Operator] One of the symbols, like \lstinline{*} and \lstinline{+}, that
represent mathematical operations.

\item[Order of operations] The rules that specify which operations
in an expression are performed first.

\item[Ordinary DE (ODE)] A differential equation in which all derivatives are taken with
respect to the same variable.

\item[Output variable] A variable in a function that is used to
return a value from the function to the caller.

\item[Pack] To copy values from a set of variables into a vector.

\item[Parameter] A value that appears in a model to quantify some
physical aspect of the scenario being modeled.

\item[Partial DE (PDE)] A differential equation that includes derivatives with respect to more than one variable.

\item[Phase plot] A plot that shows the state of a system as a point
in the space of possible states.

\item[Postcondition] Something that will be true when the script
completes.

\item[Precondition] Something that must be true when the script
starts, in order for it to work correctly.

\item[Projection] The component of one vector that is in the
direction of the other.

\item[Prompt] The symbols the interpreter prints to indicate that it's
waiting for the user to type a command.

\item[Range] A matrix of values specified with the colon operator, for example, \lstinline{1:5}.

\item[Recurrently] A way of computing the next element of a sequence
based on previous elements.

\item[Reduce] A way of processing the elements of a vector and
generating a single value, for example, the sum of the elements.

\item[Relative error] The difference between an approximation and
an exact answer, expressed as a fraction or percentage of the exact
answer.

\item[Row vector] A matrix that has only one row.

\item[Scaffolding] Code you write to help you program or debug, but
which is not part of the finished program.

\item[Scalar] A single value.

\item[Scientific notation] A format for typing and displaying large
and small numbers, e.g., \lstinline{3.0e8}, which represents $3.0 \times 10^8$
or 300,000,000.

\item[Script] An M-file that contains a sequence of MATLAB commands.

\item[Search] A way of processing a vector by examining the
elements in order until one is found that has the desired property.

\item[Search path] The list of folders where MATLAB looks for
M-files.

\item[Sequence] A set of numbers that correspond
to the positive integers, in mathematics.

\item[Series] The sum of the elements in a sequence, in mathematics.

\item[Shadow] A kind of name collision in which a new definition
causes an existing definition to become invisible.  In MATLAB,
variable names can shadow built-in functions (with hilarious results).

\item[Side effect] An effect, like modifying the workspace, that
is not the primary purpose of a function or script.

\item[Signature] The first line of a function definition, which
specifies the names of the function, the input variables, and the
output variables.

\item[Silent function] A function that doesn't display anything,
generate a figure, or have any other side effects.

\item[Spatial vector] A value that represents a
multidimensional physical quantity like position, velocity,
acceleration, or force.

\item[State] If a system can be described by a set of variables,
the values of those variables are called the state of the system.

\item[String] A value that consists of a sequence of characters.

\item[System of equations] A collection of equations written in terms of
the same set of variables.

\item[Target] The variable on the left side of an assignment statement.

\item[Time step] The interval in time between successive estimates
in the numerical solution of a differential equation.

\item[Top-level function]  The first function in an M-file;
it's the only \linebreak function that can be called from the Command
Window or from \mbox{another} file.

\item[Trajectory] A path in a phase plot that shows how the state of
a system changes over time.

\item[Transpose] An operation that transforms the rows of a matrix
into columns (and the other way around).

\item[Unit vector] A vector with norm 1, used to indicate
direction.

\item[Universal quantification] A logical condition that expresses
the idea that all elements of a set have a certain property.

\item[Unpack] To copy the elements of a vector into a set of variables.

\item[Value] The result of a computation, most often a number, string, or matrix.

\item[Variable] A named value.

\item[Vector] A sequence of values.

\item[Workspace] A set of variables and their values.

\item[Zero (of a function)] An argument that makes the value of a function $0$.

\end{description}

